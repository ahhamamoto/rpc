\documentclass[12pt,a4paper]{article}

\usepackage[utf8]{inputenc}
\usepackage[brazil]{babel}
\usepackage{indentfirst}
\usepackage{moreverb}
\usepackage{listings}
\usepackage{color}

\definecolor{dkgreen}{rgb}{0,0.6,0}
\definecolor{gray}{rgb}{0.5,0.5,0.5}
\definecolor{mauve}{rgb}{0.58,0,0.82}

\lstset{frame=tb,
	language=Python,
	aboveskip=3mm,
	belowskip=3mm,
	showstringspaces=false,
	columns=flexible,
	basicstyle={\small\ttfamily},
	numbers=none,
	numberstyle=\tiny\color{gray},
	keywordstyle=\color{blue},
	commentstyle=\color{dkgreen},
	stringstyle=\color{mauve},
	breaklines=true,
	breakatwhitespace=true
	tabsize=3
}

\renewcommand{\lstlistingname}{Código}

\author{Anderson Hiroshi Hamamoto \\ Natan de Almeida Laverde}
\date{}
\title{Agenda Eletrônica com RPC}

\begin{document}

\maketitle

\section{Introdução ao RPC}

O \textit{RPC(Remote Procedure Call)}\cite{rfc1050} define um protocolo para execução de procedimentos
remotos em computadore ligados em rede\cite{penta2}. O RPC é uma técnica muito poderosa para
criar aplicações no estilo cliente-servidor distribuídas\cite{marshall}. O RPC consegue deixar
transparente pro programador a interação entre a parte cliente e a parte servidor do 
sistema, assim, o servidor e o cliente podem estar ou não na mesma máquina, compartilhando
o mesmo espaço físico, ou podem estar conectados por meio de um servidor. 

Um exemplo de aplicação do RPC é o NFS(Network File System)\cite{rfc1094}.
Esse sistema provedencia um método transparente para para acesso remoto de dados 
compartilhados sobre a rede.

\section{Tutorial de RPC}

Nesse breve tutorial será feito uma aplicação que usa o RPC e retorna a soma e a subtração 
entre dois números.

Primeiro deve ser feito um arquivo com a extensão .x, o qual será usado para gerar os stubs. 
Stub é a parte responsável do sistema pra fazer com que a comunicaçnao das partes funcionam
corretamente, garantir que o cliente e o servidor consigam se comunicar. Nesse caso faremos
um arquivo calcula.x.

\lstset{language=C}
\begin{lstlisting}[caption=calcula.x]
	struct operandos {
	       int x;
	       int y;
	};

	program PROG {
		version VERSAO {
			int ADD(operandos) = 1;
			int SUB(operandos) = 2;
		} = 100;
	} = 55555555;
\end{lstlisting}

A struct é usada para passar parâmetros entre o cliente e o servidor. O PROG e a VERSAO
são usadas como identificadores pelo RPC e o ADD e o SUB são as funções que o cliente pode
requisitar do servidor com um identificador.

Agora é necessário usar o ``rpcgen'' para gerar os arquivos necessários. No terminal 
coloque o comando ``rpcgen -a calcula.x'', isso vai gerar os arquivos:

\begin{itemize}
	\item calcula.h - cabeçalho da interface das funções e da struct que são usadas no programa
	\item calcula\_svc.c - stub do servidor
	\item calcula\_clnt.c - stub do cliente
	\item calcula\_client.c - programador que define o que acontece no lado do cliente
	\item calcula\_server.c - programador define o que o que o servidor faz com a requisição do cliente
	\item calcula.xdr - filtros de representações de dados externos(eXternal Data Representation)\cite{rfc1014}
\end{itemize}

Os arquivos que precisamos nos preocupar são o calcula\_client.c e o calcula\_server.c, que são
o que programador implementa para fazer uso do RPC. No arquivo calcula\_client.c é colocado tudo
o que o programador quer que ocorra no lado do cliente e calcula\_server.c é colocado o que 
o servidor executa.

No nosso exemplo tanto o servidor como cliente não tem muito o que ser feito.

\begin{lstlisting}[caption=client\_client.c]
	int add(CLIENT *clnt, int x, int y) {
	  operandos ops;
	  int *result;

	  ops.x = x;
	  ops.y = y;

	  result = add_100(&ops, clnt);
	  if (result == NULL) {
	    fprintf(stderr, "Problema na chamada RPC\n");
	    exit(0);
	  }
	  return (*result);
	}

	int sub(CLIENT *clnt, int x, int y) {
	  operandos ops;
	  int *result;

	  ops.x = x;
	  ops.y = y;

	  result = sub_100(&ops, clnt);
	  if (result == NULL) {
	    fprintf(stderr, "Problema na chamada RPC\n");
	    exit(0);
	  }
	  return (*result);
	}
	
	int main(int argc, char **argv) {
	  CLIENT *clnt;
	  int x, y;

	  if (argc != 4) {
	    fprintf(stderr, "Usagem errada\n");
	    exit(0);
	  }
	  clnt = clnt_create(argv[1], PROG, VERSAO, "udp");
	  if (clnt == (CLIENT *) NULL) {
	    clnt_pcreateerror(argv[1]);
	    exit(1);
	  }
	  x = atoi(argv[2]);
	  y = atoi(argv[3]);
	  printf("%d + %d = %d\n",x,y, add(clnt,x,y));
	  printf("%d - %d = %d\n",x,y, sub(clnt,x,y));

	  return 0;
	}
\end{lstlisting}

\begin{lstlisting}[caption=calcula\_server.c]
	int * add_100_svc(operandos *argp, struct svc_req *rqstp) {
	  static int  result;

	  printf("Requisicao de adicao para %d e %d\n", argp->x, argp->y);
	  result = argp->x + argp->y;
  
	  return(&result);
	}

	int * sub_100_svc(operandos *argp, struct svc_req *rqstp) {
	  static int  result;
  
	  printf("Requisicao de subtracao para %d e %d\n", argp->x, argp->y);
	  result = argp->x - argp->y;
  
	  return(&result);
	}
\end{lstlisting}

Para compilar o programa feito, o cliente e o servidor devem ser compilados separadamente:
\begin{itemize}
	\item servidor: ``gcc -o servidor calcula.h calcula\_svc.c calcula\_server.c calcula\_xdr.c''
	\item cliente: ``gcc -o cliente calcula.h calcula\_clnt.c calcula\_client.c calcula\_xdr.c''
\end{itemize}

Depois basta executar o servidor, e a máquina que irá rodar o servidor deve estar com o portmapper
ativado, no Linux e no Mac Os tem programa chamado ``rpcbind'' que também pode ser usado. Para executar
o cliente tem que passar o \textit{host}, que no caso é o servidor pra poder executar o programa.
Para executar o cliente também é necessário passar como argumento para o programa os números
que o usuário quer que faça a soma e a subtração.

\section{Projeto da Agenda Eletrônica}

A agenda eletrônica envolve o uso de um banco de dados para armazenar entradas de
uma agenda. Essa agenda armazena nome, endereço e telefone.

O banco de dados que escolhemos usar para esse projeto é o MySQL Community Server\cite{mysql}.
Tem uma base de dados com o nome de agenda, dentro desse banco tem uma tabela com o nome de
``agendarpc''. Essa tabela contém os campos nome, endereço e telefone, em que o nome é a chave
primária da tabela.

\lstset{language=SQL}
\begin{lstlisting}[caption=Comando no MySQL usado para criar a tabela]
	CREATE TABLE agendarpc(
		name VARCHAR(100) PRIMARY KEY,
		addr VARCHAR(100),
		phone VARCHAR(100)
	);
\end{lstlisting}

\lstset{language=C}
\begin{lstlisting}[caption=Struct para organizar os dados]
	struct entry {
		char *name;
		char *addr;
		char *phone;
	};
\end{lstlisting}

O nosso programa tem 5 funções básicas: inicializar, inserir, alterar, remover e consultar.
A função inicializar deleta todas as entradas da tabela do banco de dados e inserir todas 
as entradas no arquivo ``entrada.txt''.

\begin{lstlisting}[caption=Formato do arquivo de entrada para a função inicializar]
	nome 1
	
	endereco 1
	
	telefone 1
	
	nome 2
	
	endereco 2
	
	telefone 2
\end{lstlisting}

A função inserir tem como objetivo inserir um conjunto de dados (nome, endereço e telefone).
Para implementar essa função, a query é montada em uma string que então é passada para o banco
de dados. A função de alterar, ela modifica os dados que estão na tabela do banco de dados.
Remover retira uma tupla da tabela do banco de dados. A função de consultar retorna uma das 
tuplas usando o nome para buscar.

Abaixo tem as funções de inserir, alterar e remover, antes de aplicá-los no RPC.

\lstset{language=C}
\begin{lstlisting}
	typedef struct entry entry;

	struct entry {
	  char *name;
	  char *addr; 
	  char *phone;
	};
	
	void insert(MYSQL *conn, entry *data) {
	  int n;
	  char query[100];
	  n = sprintf(query, "INSERT INTO teste VALUES('%s', '%s', '%s')", data->name, data->addr, data->phone);
	  if (!mysql_query(conn, query)) 
			printf("%s\n", mysql_error(conn));
	}

	int alter(MYSQL *conn, entry *old, entry *new) {
	  int n;
	  char query[100];
	  n = sprintf(query, "UPDATE teste SET name='%s', address='%s', phone='%s' WHERE name='%s'", new->name, new->addr, new->phone, old->name);
	  printf("%s\n", query);
	  if (!mysql_query(conn, query)) 
			printf("%s\n", mysql_error(conn));
	} 

	int delete(MYSQL *conn, entry *data) {
	  int n;
	  char query[100];
	  n = sprintf(query, "DELETE FROM teste WHERE name='%s'", data->name);
	  if (!mysql_query(conn, query)) 
			printf("%s\n", mysql_error(conn));
	}
\end{lstlisting}

\bibliography{reference}
\bibliographystyle{unsrt}

\end{document}















